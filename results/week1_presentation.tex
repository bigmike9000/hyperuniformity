% !TEX program = pdflatex
\documentclass[aspectratio=169, 11pt]{beamer}

% ── Theme ──────────────────────────────────────────────────────────────
\usetheme{Madrid}
\usecolortheme{seahorse}
\setbeamertemplate{navigation symbols}{}
\setbeamertemplate{footline}{%
  \leavevmode\hbox{%
    \begin{beamercolorbox}[wd=.33\paperwidth,ht=2.5ex,dp=1ex,center]{author in head/foot}%
      \usebeamerfont{author in head/foot}\insertshortauthor
    \end{beamercolorbox}%
    \begin{beamercolorbox}[wd=.34\paperwidth,ht=2.5ex,dp=1ex,center]{title in head/foot}%
      \usebeamerfont{title in head/foot}\insertshorttitle
    \end{beamercolorbox}%
    \begin{beamercolorbox}[wd=.33\paperwidth,ht=2.5ex,dp=1ex,right]{date in head/foot}%
      \usebeamerfont{date in head/foot}\insertframenumber{} / \inserttotalframenumber\hspace*{2ex}
    \end{beamercolorbox}}%
  \vskip0pt%
}

% ── Packages ───────────────────────────────────────────────────────────
\usepackage{amsmath,amssymb}
\usepackage{graphicx}
\usepackage{booktabs}
\usepackage{xcolor}
\usepackage{tikz}
\usepackage{pgfpages}
\usepackage{bookmark}

% ── Speaker notes ─────────────────────────────────────────────────────
% pgfpages notes mode produces cosmetic "duplicate destination" warnings
% from pdfTeX — these do not affect the output and can be safely ignored.
\setbeameroption{show notes on second screen=right}
% For notes-only PDF:   \setbeameroption{show only notes}
% For slides-only PDF:  comment out the setbeameroption line above

% ── Paths ──────────────────────────────────────────────────────────────
\graphicspath{{./}}

% ── Custom colors ──────────────────────────────────────────────────────
\definecolor{fibgreen}{HTML}{2ca02c}
\definecolor{silvpurp}{HTML}{9467bd}
\definecolor{bronred}{HTML}{d62728}
\definecolor{resultbg}{HTML}{F0FFF0}

% ── Title ──────────────────────────────────────────────────────────────
\title[Hyperuniformity of 1D Point Patterns]{%
  Hyperuniformity of 1D Point Patterns}
\subtitle{Week 1 Progress}
\author{Michael Fang}
\institute{Princeton University}
\date{February 25, 2026}

% ======================================================================
\begin{document}

% ── Slide 1: Title ────────────────────────────────────────────────────
\begin{frame}
  \titlepage
\end{frame}
\note{
  Week 1 progress update on the 1D hyperuniformity project.
}

% ======================================================================
\section{Background}
% ======================================================================

% ── Slide 2: Background ──────────────────────────────────────────────
\begin{frame}{Background}
  A point pattern is \textbf{hyperuniform} if density fluctuations
  are suppressed relative to random
  {\small(Torquato \& Stillinger, 2003)}:
  \begin{equation*}
    \sigma^2(R) \sim R^{d-\alpha},
    \qquad
    S(k) \sim |k|^{\alpha} \;\text{as}\; k\to 0
  \end{equation*}

  \medskip
  \textbf{Three chains} built from substitution rules on tiles $S{=}1$, $L{=}\mu$:

  \smallskip
  \begin{table}
    \centering\small
    \begin{tabular}{lccc}
      \toprule
      \textbf{Chain} & \textbf{Rule} & \textbf{Metallic mean $\mu$} & $\boldsymbol{\rho}$\\
      \midrule
      Fibonacci & $S\!\to\!L,\;L\!\to\!LS$   & $\tau\approx1.618$ & 0.724\\
      Silver    & $S\!\to\!L,\;L\!\to\!LLS$  & $\mu_2\approx2.414$ & 0.500\\
      Bronze    & $S\!\to\!L,\;L\!\to\!LLLS$ & $\mu_3\approx3.303$ & 0.361\\
      \bottomrule
    \end{tabular}
  \end{table}

  \medskip
  \textbf{Goal:} numerically verify the eigenvalue prediction
  \fcolorbox{black}{resultbg}{$\;\alpha = 3\;$}
  {\small(O\u{g}uz et al., 2019)} for all three chains,
  using diffusion spreadability {\small(Torquato, 2021)}.
\end{frame}
\note{
  Hyperuniformity: Torquato \& Stillinger (2003).
  Eigenvalue formula: O\u{g}uz et al.\ (2019), Acta Cryst.\ A 75.
  Spreadability method: Torquato (2021), Phys.\ Rev.\ E 104.
}

% ======================================================================
\section{Validation}
% ======================================================================

% ── Slide 3: Poisson Benchmark ───────────────────────────────────────
\begin{frame}{Code Validation: Poisson Benchmark}
  \centering
  \includegraphics[width=0.88\linewidth]{fig1_poisson_benchmark.png}

  \smallskip
  {\small Exact result $\sigma^2=2\rho R$ reproduced to 1.1\% mean error
   --- variance algorithm validated.}
\end{frame}
\note{
  100 independent Poisson patterns, $N{=}10{,}000$.
  This validates the sliding-window algorithm used for all subsequent measurements.
}

% ======================================================================
\section{Results}
% ======================================================================

% ── Slide 4: Bounded Variance ────────────────────────────────────────
\begin{frame}{Bounded Variance --- Class I Confirmed}
  \begin{columns}[T]
    \begin{column}{0.58\textwidth}
      \centering
      \includegraphics[width=\linewidth]{fig2_bounded_variance.png}
    \end{column}
    \begin{column}{0.38\textwidth}
      \begin{table}
        \centering\small
        \begin{tabular}{lcc}
          \toprule
          \textbf{Pattern} & $\boldsymbol{\bar{\Lambda}}$ & \textbf{Exact}\\
          \midrule
          Lattice   & 0.165 & $1/6$\\
          Fibonacci & 0.200 & ---\\
          Silver    & 0.250 & ---\\
          Bronze    & 0.282 & ---\\
          \bottomrule
        \end{tabular}
      \end{table}

      \medskip\small
      \begin{itemize}
        \item All curves bounded $\Rightarrow$ Class~I
        \item Lattice $\bar{\Lambda}=1/6$ matches theory
        \item $\bar{\Lambda}$ increases with $\mu$
      \end{itemize}
    \end{column}
  \end{columns}
\end{frame}
\note{
  Lattice $\bar{\Lambda}=1/6$ exact (Torquato \& Stillinger, 2003).
  Fibonacci $\bar{\Lambda}=0.201$ reported numerically in Zachary \& Torquato (2009);
  our 0.200 matches to 0.5\%.
}

% ── Slide 5: Projection Comparison ──────────────────────────────────
\begin{frame}{Projection Method: Class I vs.\ Class II}
  \centering
  \includegraphics[width=0.85\linewidth]{fig4_projection_comparison.png}

  \smallskip
  {\small Left: ideal strip $\omega=\tau$ (Class~I, bounded).
   Right: $\omega=0.9\tau$ (Class~II, logarithmic growth).
   Two independent generation methods confirm the same $\bar{\Lambda}$.}
\end{frame}
\note{
  The cut-and-project method provides an independent realization.
  Any deviation from ideal strip width degrades Class~I to Class~II.
}

% ── Slide 6: Spreadability Method ────────────────────────────────────
\begin{frame}{Spreadability Method}
  \textbf{Problem:} $S(k)$ has dense Bragg peaks
  $\;\Longrightarrow\;$ can't read off $\alpha$ directly
  {\small(Torquato, 2021)}.

  \bigskip
  \textbf{Solution:} embed points in a two-phase medium
  {\small(Torquato, 2002)} and measure
  diffusion spreadability {\small(Torquato, 2021)}:
  \begin{enumerate}
    \item Decorate each point with a solid rod (packing fraction $\phi_2=0.35$)
    \item Compute spectral density $\tilde{\chi}_V(k) = \rho\,|\tilde{m}(k)|^2 S(k)$
    \item Evaluate excess spreadability $E(t)$ via Gaussian-smoothed sum
    \item Extract $\alpha$ from long-time power-law decay:
      $E(t)\sim t^{-(1+\alpha)/2}$
  \end{enumerate}

  \bigskip
  The Gaussian kernel $e^{-k^2 Dt}$ naturally smooths over the dense
  Bragg peaks, revealing the underlying $\alpha$
  {\small(Torquato, 2021; Kim \& Torquato, 2024)}.
\end{frame}
\note{
  Two-phase media: Torquato (2002), Random Heterogeneous Materials, Ch.~2.
  Spreadability method: Torquato (2021), Phys.\ Rev.\ E 104, 054102.
  Applied to quasicrystals: Kim \& Torquato (2024).
  Packing fraction $\phi_2=0.35$ follows from Hitin-Bialus (2024).
}

% ── Slide 7: Main Result ─────────────────────────────────────────────
{
\setbeamercolor{frametitle}{bg=resultbg, fg=black}
\begin{frame}{\texorpdfstring{%
  Main Result: $\alpha=3$ for All Three Chains}{%
  Main Result: alpha=3 for All Three Chains}}
  \begin{columns}[T]
    \begin{column}{0.58\textwidth}
      \centering
      \includegraphics[width=\linewidth]{fig8_alpha_extraction.png}
    \end{column}
    \begin{column}{0.38\textwidth}
      \begin{table}
        \centering\small
        \begin{tabular}{lcc}
          \toprule
          \textbf{Pattern} & $\boldsymbol{\alpha}$ & \textbf{Exp.}\\
          \midrule
          Poisson & 0.003 & 0\\
          \textcolor{fibgreen}{Fibonacci}
            & \textcolor{fibgreen}{\textbf{3.049}} & 3\\
          \textcolor{silvpurp}{Silver}
            & \textcolor{silvpurp}{\textbf{2.992}} & 3\\
          \textcolor{bronred}{Bronze}
            & \textcolor{bronred}{\textbf{2.987}} & 3\\
          \bottomrule
        \end{tabular}
      \end{table}

      \medskip\small
      Errors: Fib 1.6\%, Ag 0.3\%, Br 0.4\%.
      Linear fit over $t\in[10^2,10^5]$.

      \medskip
      \fcolorbox{black}{resultbg}{\parbox{0.9\linewidth}{\centering\small
        \textbf{$\alpha=3$ confirmed},\\
        matching O\u{g}uz et al.\ (2019).
      }}
    \end{column}
  \end{columns}
\end{frame}
}
\note{
  Central result: all three chains give $\alpha\approx 3$,
  matching the eigenvalue prediction $\alpha=1-2\ln|\lambda_2|/\ln\lambda_1$
  from O\u{g}uz et al.\ (2019). Poisson baseline confirms $\alpha\approx 0$.
}

\end{document}
