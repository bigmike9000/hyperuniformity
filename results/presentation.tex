% !TEX program = pdflatex
\documentclass[aspectratio=169, 11pt]{beamer}

% ── Theme ──────────────────────────────────────────────────────────────
\usetheme{Madrid}
\usecolortheme{seahorse}
\setbeamertemplate{navigation symbols}{}
\setbeamertemplate{footline}{%
  \leavevmode\hbox{%
    \begin{beamercolorbox}[wd=.33\paperwidth,ht=2.5ex,dp=1ex,center]{author in head/foot}%
      \usebeamerfont{author in head/foot}\insertshortauthor
    \end{beamercolorbox}%
    \begin{beamercolorbox}[wd=.34\paperwidth,ht=2.5ex,dp=1ex,center]{title in head/foot}%
      \usebeamerfont{title in head/foot}\insertshorttitle
    \end{beamercolorbox}%
    \begin{beamercolorbox}[wd=.33\paperwidth,ht=2.5ex,dp=1ex,right]{date in head/foot}%
      \usebeamerfont{date in head/foot}\insertframenumber{} / \inserttotalframenumber\hspace*{2ex}
    \end{beamercolorbox}}%
  \vskip0pt%
}

% ── Packages ───────────────────────────────────────────────────────────
\usepackage{amsmath,amssymb}
\usepackage{graphicx}
\usepackage{booktabs}
\usepackage{xcolor}
\usepackage{tikz}
\usepackage{pgfpages}

% ── Enable speaker notes (comment out for handout without notes) ──────
\setbeameroption{show notes on second screen=right}
% For a notes-only PDF:  \setbeameroption{show only notes}
% For slides-only PDF:   comment out the setbeameroption line above

% ── Paths ──────────────────────────────────────────────────────────────
% Figures are in the same directory as this .tex file.
% If compiling from a different directory, adjust the path below.
\graphicspath{{./}}

% ── Custom colors ──────────────────────────────────────────────────────
\definecolor{fibgreen}{HTML}{2ca02c}
\definecolor{silvpurp}{HTML}{9467bd}
\definecolor{bronred}{HTML}{d62728}
\definecolor{resultbg}{HTML}{F0FFF0}

% ── Title ──────────────────────────────────────────────────────────────
\title[Hyperuniformity in 1D Quasicrystals]{%
  Hyperuniformity in 1D Quasiperiodic\\Point Patterns}
\subtitle{Numerical Extraction of the Scaling Exponent\\
  via Diffusion Spreadability}
\author{Presented to Professor Salvatore Torquato}
\institute{Princeton University}
\date{\today}

% ======================================================================
\begin{document}

% ── Title Slide ───────────────────────────────────────────────────────
\begin{frame}
  \titlepage
\end{frame}
\note{
  This presentation covers four completed phases of the 1D hyperuniformity
  project: code validation, pattern generation, real-space variance analysis,
  and reciprocal-space spreadability analysis.  The main result is the
  numerical confirmation that $\alpha=3$ universally for all metallic-mean
  substitution tilings.
}

% ── Outline ───────────────────────────────────────────────────────────
\begin{frame}{Outline}
  \tableofcontents
\end{frame}

% ======================================================================
\section{Background}
% ======================================================================

% ── Slide 1 ───────────────────────────────────────────────────────────
\begin{frame}{What Is Hyperuniformity?}
  A point pattern is \textbf{hyperuniform} if local number fluctuations
  are \emph{suppressed} relative to random:
  \begin{equation*}
    \sigma^2(R) \sim R^{d-\alpha},
    \qquad
    S(k) \sim |k|^{\alpha} \;\text{as}\; k\to 0
  \end{equation*}

  \medskip
  \begin{table}
    \centering\small
    \begin{tabular}{llll}
      \toprule
      \textbf{Class} & \textbf{Exponent} & \textbf{Variance growth} & \textbf{Examples}\\
      \midrule
      I   & $\alpha>1$      & $R^{d-1}$ (surface area) & crystals, quasicrystals\\
      II  & $\alpha=1$      & $R^{d-1}\ln R$           & period-doubling chains\\
      III & $0<\alpha<1$    & $R^{d-\alpha}$           & certain aperiodic chains\\
      \bottomrule
    \end{tabular}
  \end{table}

  \bigskip
  \textbf{Goal:} numerically extract $\alpha$ for three 1D quasicrystal
  families and verify $\alpha=3$.

  \medskip
  \textbf{Challenge:} dense Bragg peaks make direct $S(k)$ measurement
  impossible $\;\Longrightarrow\;$ use \emph{diffusion spreadability}
  (Torquato, 2021).
\end{frame}
\note{
  Hyperuniformity was introduced by Torquato \& Stillinger (2003).
  The key idea: for a Poisson process $\sigma^2 \sim R^d$ (volume scaling),
  while hyperuniform systems grow \emph{slower}.  The exponent $\alpha$
  classifies how much slower.  Class~I ($\alpha>1$) is the strongest form,
  with variance scaling like the surface area of the window.

  The ``measurement problem'' for quasicrystals motivates the two-phase
  media / spreadability approach that is the heart of Phase~4.
}

% ── Slide 2 ───────────────────────────────────────────────────────────
\begin{frame}{The Three Metallic-Mean Quasicrystal Chains}
  \begin{table}
    \centering\small
    \begin{tabular}{lcccc}
      \toprule
      \textbf{Chain} & \textbf{Rule} & \textbf{Matrix $\mathbf{M}$}
        & \textbf{Metallic mean $\mu$} & $\boldsymbol{\rho}$\\
      \midrule
      Fibonacci & $S\!\to\!L,\;L\!\to\!LS$
        & $\bigl(\begin{smallmatrix}0&1\\1&1\end{smallmatrix}\bigr)$
        & $\tau\approx1.618$ & 0.724\\[4pt]
      Silver & $S\!\to\!L,\;L\!\to\!LLS$
        & $\bigl(\begin{smallmatrix}0&1\\1&2\end{smallmatrix}\bigr)$
        & $\mu_2\approx2.414$ & 0.500\\[4pt]
      Bronze & $S\!\to\!L,\;L\!\to\!LLLS$
        & $\bigl(\begin{smallmatrix}0&1\\1&3\end{smallmatrix}\bigr)$
        & $\mu_3\approx3.303$ & 0.361\\
      \bottomrule
    \end{tabular}
  \end{table}

  \medskip
  \textbf{Theoretical prediction} (eigenvalue formula):
  \begin{equation*}
    \alpha = 1 - \frac{2\ln|\lambda_2|}{\ln\lambda_1}
  \end{equation*}
  Since $\det(\mathbf{M})=-1$ for all three $\;\Longrightarrow\;
  |\lambda_2|=1/\lambda_1$ $\;\Longrightarrow\;$
  \fcolorbox{black}{resultbg}{$\;\alpha = 3\;$ \textbf{(universal)}}

  \medskip
  Tile lengths: $S=1$, $L=\mu$.  Production chains: $N\sim10^7$ tiles.
\end{frame}
\note{
  Each chain is defined by a $2\times 2$ substitution matrix acting on
  tile counts $[n_S, n_L]$.  The eigenvalues determine both the growth rate
  (largest eigenvalue $\lambda_1 = \mu$) and the hyperuniformity exponent
  via the second eigenvalue.

  Because $\det(\mathbf{M})=-1$ for all metallic-mean matrices,
  $|\lambda_2| = 1/\mu$, giving $\alpha = 1 + 2 = 3$ universally.
  This is a purely algebraic result --- our job is to verify it numerically.
}

% ======================================================================
\section{Phase 1: Code Validation}
% ======================================================================

% ── Slide 3 ───────────────────────────────────────────────────────────
\begin{frame}{Figure 1: Poisson Variance Benchmark}
  \centering
  \includegraphics[width=0.92\linewidth]{fig1_poisson_benchmark.png}

  \smallskip
  {\small Exact result $\sigma^2=2\rho R$ reproduced to 1.8\% mean error
   $\;\Longrightarrow\;$ sliding-window algorithm \textbf{validated}.}
\end{frame}
\note{
  \textbf{What this shows:} We generate 40 independent Poisson point
  patterns ($N{=}10{,}000$, $\rho{=}1$) and compute $\sigma^2(R)$ via
  binary-search sliding windows. \\[6pt]
  \textbf{Left panel:} Computed variance (black dots $\pm 2\sigma$)
  vs.\ exact theory $\sigma^2=2\rho R$ (red dashed). \\[6pt]
  \textbf{Right panel:} Relative error at each $R$, all below 5\%.
  Green box shows mean error = 1.8\%. \\[6pt]
  \textbf{Why it matters:} This same algorithm is used for all subsequent
  variance computations.  Validating it against the Poisson exact result
  ensures we can trust the quasicrystal measurements.
}

% ======================================================================
\section{Phase 2: Pattern Generation}
% ======================================================================

% ── Slide 4 ───────────────────────────────────────────────────────────
\begin{frame}{Two Independent Generation Methods}
  \begin{columns}[T]
    \begin{column}{0.48\textwidth}
      \textbf{Method 1: Substitution}
      \begin{itemize}
        \item Iterate substitution rules from seed $L$
        \item 34 iterations $\Rightarrow$ Fibonacci $N{=}14{,}930{,}352$
        \item Generates $10^7$ tiles in $\sim$1\,s
        \item Exactly 2 distinct spacings (verified)
      \end{itemize}
    \end{column}
    \begin{column}{0.48\textwidth}
      \textbf{Method 2: Cut-and-Project}
      \begin{itemize}
        \item Embed $\mathbb{Z}^2$ lattice
        \item Strip of width $\omega$ along slope $1/\mu$
        \item Project interior points onto 1D line
        \item $\omega=\mu$: Class~I;\; $\omega\neq\mu$: Class~II
      \end{itemize}
    \end{column}
  \end{columns}

  \bigskip
  Both methods produce the \textbf{same quasicrystal} (up to rescaling),
  allowing cross-validation of $\bar{\Lambda}$.

  \medskip
  \begin{table}
    \centering\small
    \begin{tabular}{lcc}
      \toprule
      \textbf{Chain} & \textbf{Density $\rho$} & \textbf{Spacing ratio $L/S$}\\
      \midrule
      Fibonacci & 0.7236 & $\tau\approx 1.618$\\
      Silver    & 0.5000 & $\mu_2\approx 2.414$\\
      Bronze    & 0.3613 & $\mu_3\approx 3.303$\\
      \bottomrule
    \end{tabular}
  \end{table}
\end{frame}
\note{
  The substitution method is conceptually simple: start with $L$, apply the
  rule, repeat.  The cut-and-project method provides an independent
  realization at a different density, which is crucial for verifying the
  rescaling invariance of $\bar{\Lambda}$.

  Key point: substitution gives $\rho\approx 0.72$ for Fibonacci while
  projection gives $\rho\approx 0.85$, yet both yield the same
  $\bar{\Lambda}=0.200$.
}

% ======================================================================
\section{Phase 3: Real-Space Analysis}
% ======================================================================

% ── Slide 5 ───────────────────────────────────────────────────────────
\begin{frame}{Figure 2: Bounded Variance --- Class I Confirmed}
  \centering
  \includegraphics[width=0.75\linewidth]{fig2_bounded_variance.png}
\end{frame}
\note{
  \textbf{What this shows:} Number variance $\sigma^2(R)$ for the integer
  lattice (top-left) and three quasicrystal chains.  Each curve: 30,000
  random windows at 1,000 $R$-values. \\[6pt]
  \textbf{Key observation:} All four curves are \emph{bounded and
  oscillating} --- they do not grow with $R$.  This is the defining
  signature of Class~I hyperuniformity. \\[6pt]
  \textbf{Red dashed lines:} $\bar{\Lambda}$ (surface-area coefficient).
  Lattice: $\bar{\Lambda}\approx 1/6$.  Values increase monotonically
  with $\mu$: Fibonacci (0.200) $<$ Silver (0.250) $<$ Bronze (0.282).
  \\[6pt]
  \textbf{Interpretation:} More ``aperiodic'' chains (higher $\mu$)
  have larger fluctuation amplitudes, but all remain Class~I.
}

% ── Slide 6: Lambda-bar table ─────────────────────────────────────────
\begin{frame}{Surface-Area Coefficient $\bar{\Lambda}$}
  \begin{table}
    \centering
    \begin{tabular}{lccc}
      \toprule
      \textbf{Pattern} & $\boldsymbol{N}$ \textbf{(tiles)}
        & $\boldsymbol{\bar{\Lambda}}$ & \textbf{vs Lattice}\\
      \midrule
      Integer Lattice          & 100,000     & 0.165 ($\approx 1/6$) & reference\\
      Fibonacci (substitution) & 14,930,352  & 0.200 & 1.21$\times$\\
      Silver (substitution)    & 22,619,537  & 0.250 & 1.51$\times$\\
      Bronze (substitution)    & 21,932,293  & 0.282 & 1.71$\times$\\
      \midrule
      Fibonacci (projection)   & 220,161     & 0.200 & 1.21$\times$\\
      \bottomrule
    \end{tabular}
  \end{table}

  \bigskip
  \begin{itemize}
    \item $\bar{\Lambda}$ increases monotonically with metallic mean $\mu$
    \item Substitution vs.\ projection agree to \textbf{0.1\%} for Fibonacci
    \item[$\Rightarrow$] $\bar{\Lambda}$ is \textbf{density-independent}
      (rescaling invariance confirmed)
  \end{itemize}
\end{frame}
\note{
  The projection method generates Fibonacci at $\rho\approx 0.85$ while
  substitution gives $\rho\approx 0.72$.  Despite the density difference,
  $\bar{\Lambda}$ agrees to 0.1\%.  Under rescaling $x\to ax$, we have
  $\sigma'^2(R)=\sigma^2(R/a)$ whose long-range average is unchanged.
  This numerical verification confirms the theoretical expectation.
}

% ── Slide 7 ───────────────────────────────────────────────────────────
\begin{frame}{Figure 3: Hyperuniformity Test --- $\sigma^2(R)/R \to 0$}
  \centering
  \includegraphics[width=0.72\linewidth]{fig3_hyperuniformity_test.png}
\end{frame}
\note{
  \textbf{What this shows:} The ratio $\sigma^2(R)/R$ on a log scale
  for all three chains.  For Poisson, this ratio equals
  $2\rho \approx 1.45 = \text{const}$ (gray dashed line at top).
  For any hyperuniform pattern, it must decay to zero. \\[6pt]
  \textbf{Key observation:} On the log $y$-axis, the decay from
  $\sim 10^{-1}$ to $\sim 10^{-3}$ is clearly visible.  All three
  chains show power-law--like decay, confirming that number fluctuations
  grow much slower than the window size. \\[6pt]
  \textbf{Why it matters:} This is the simplest direct real-space test
  of hyperuniformity.  The three orders-of-magnitude gap between
  the Poisson constant and the quasicrystal curves at large $R$ is
  the signature of suppressed fluctuations.
}

% ── Slide 8 ───────────────────────────────────────────────────────────
\begin{frame}{Figure 4: Projection --- Class I vs.\ Class II}
  \centering
  \includegraphics[width=0.88\linewidth]{fig4_projection_comparison.png}
\end{frame}
\note{
  \textbf{Left panel (Class~I):} Ideal strip width $\omega=\tau$.
  Variance is bounded with $\bar{\Lambda}=0.200$, matching the
  substitution result to 0.1\%. \\[6pt]
  \textbf{Right panel (Class~II):} Non-ideal strip width $\omega=0.9\tau$.
  The variance envelope \emph{grows logarithmically}:
  $\sigma^2\sim 0.067\ln R + b$.  This logarithmic growth is the hallmark
  of Class~II hyperuniformity. \\[6pt]
  \textbf{Takeaway:} The hyperuniformity class is controlled by the
  projection strip width.  Any deviation from the ideal $\omega=\mu$
  degrades Class~I to Class~II.
}

% ── Slide 9 ───────────────────────────────────────────────────────────
\begin{frame}{Figure 5: Piecewise Quadratic Structure}
  \centering
  \includegraphics[width=0.68\linewidth]{fig5_piecewise_quadratic.png}
\end{frame}
\note{
  \textbf{Top-left:} Lattice variance at high resolution, overlaid with
  exact formula $\sigma^2(R)=2\{R\}(1-2\{R\})$.  Perfect agreement.
  \\[6pt]
  \textbf{Top-right:} The exact second derivative
  $d^2\sigma^2/dR^2 = -8$ is constant between breakpoints at every
  half-integer $R$.  Vertical lines mark these breakpoints.
  This confirms $\sigma^2$ is piecewise quadratic. \\[6pt]
  \textbf{Bottom-left:} Fibonacci variance shows self-similar oscillations
  at two incommensurate scales ($S{=}1$ and $L{=}\tau$). \\[6pt]
  \textbf{Bottom-right:} Fibonacci first derivative $d\sigma^2/dR$
  is piecewise linear with slope changes at quasiperiodic positions
  determined by $S$ and $L$ tile lengths --- confirming the piecewise
  quadratic structure with a dense set of breakpoints.
}

% ======================================================================
\section{Phase 4: Two-Phase Media \& Spreadability}
% ======================================================================

% ── Slide 10 ──────────────────────────────────────────────────────────
\begin{frame}{The Two-Phase Media Approach}
  \textbf{Problem:} $S(k)$ has dense Bragg peaks at irrational $k$
  $\;\Longrightarrow\;$ cannot read off $\alpha$ directly.

  \bigskip
  \textbf{Solution --- four steps:}
  \begin{enumerate}
    \item \textbf{Decorate} each point with a solid rod (half-length
      $a=\phi_2/2\rho$,\; $\phi_2=0.35$)
    \item \textbf{Compute} spectral density:\;
      $\tilde{\chi}_V(k) = \rho\,\bigl(\tfrac{2\sin ka}{k}\bigr)^{\!2} S(k)$
    \item \textbf{Evaluate} excess spreadability:\;
      $E(t) = \frac{\Delta k}{\pi\phi_2}
       \sum_n \tilde{\chi}_V(k_n)\,e^{-k_n^2 Dt}$
    \item \textbf{Extract} $\alpha$ from long-time decay:\;
      $\alpha(t) = -2\,\frac{d\ln E}{d\ln t} - 1$
  \end{enumerate}

  \bigskip
  \begin{table}
    \centering\small
    \begin{tabular}{lcccl}
      \toprule
      \textbf{Chain} & $\boldsymbol{\rho}$ & \textbf{Rod $2a$}
        & \textbf{Min gap} & \textbf{Overlap?}\\
      \midrule
      Fibonacci & 0.724 & 0.484 & 1.000 & No\\
      Silver    & 0.500 & 0.700 & 1.000 & No\\
      Bronze    & 0.361 & 0.969 & 1.000 & No\\
      \bottomrule
    \end{tabular}
  \end{table}
\end{frame}
\note{
  The two-phase media approach was introduced by Torquato (2021).
  The key insight: the Gaussian kernel $e^{-k^2 Dt}$ in the spreadability
  integral acts as a natural smoother of the Bragg peak spectrum.  At time
  $t$, it samples wavevectors $k \lesssim 1/\sqrt{Dt}$.  The power-law
  tail of the Bragg peak envelope then translates into a power-law decay
  of $E(t)$, from which $\alpha$ can be extracted.

  Non-overlap is verified: $2a < \min(\text{spacings}) = 1.0$ for all chains.
  The packing fraction $\phi_2 = 0.35$ is chosen following Hitin-Bialus
  et al.\ (2024).
}

% ── Slide 11 ──────────────────────────────────────────────────────────
\begin{frame}{Figure 6: Spectral Density Reveals $k^3$ Scaling}
  \centering
  \includegraphics[width=0.68\linewidth]{fig6_spectral_density.png}
\end{frame}
\note{
  \textbf{Top-left (Poisson):} Flat $S(k)=1$ envelope, modulated only by
  the rod form factor $|\tilde{m}(k)|^2$.  No hyperuniformity. \\[4pt]
  \textbf{Top-right (Lattice):} Bragg peaks at $k=2\pi n$.  The FFT
  computation introduces a noise floor between the true peaks (this
  does not affect results since the lattice spreadability is computed
  analytically). \\[4pt]
  \textbf{Bottom-left (Fibonacci):} Dense Bragg peaks.  The envelope of
  peak heights follows $k^3$ at small $k$ --- the spectral signature
  of $\alpha=3$. \\[4pt]
  \textbf{Bottom-right (All chains overlaid):} All three chains share the
  same $k^3$ envelope (black dashed reference line), confirming
  universality. \\[4pt]
  The $k^3$ scaling is the Fourier-space manifestation of
  $\alpha=3$ Class~I hyperuniformity.
}

% ── Slide 12 ──────────────────────────────────────────────────────────
\begin{frame}{Figure 7: Excess Spreadability --- Decay Rates}
  \centering
  \includegraphics[width=0.88\linewidth]{fig7_excess_spreadability.png}

  \smallskip
  {\small At long times: $E(t)\sim t^{-(1+\alpha)/2}$.
   More hyperuniform $=$ faster equilibration.}
\end{frame}
\note{
  \textbf{Main panel (left):} Log-log plot of excess spreadability $E(t)$
  for Poisson and the three quasicrystals. \\[4pt]
  \textbf{Gray (Poisson):} Slow decay $\sim t^{-1/2}$ ($\alpha=0$). \\[4pt]
  \textbf{Green/Purple/Red (quasicrystals):} Steeper decay $\sim t^{-2}$
  ($\alpha=3$).  The gold shaded region marks the measurement window
  $[10^2, 10^5]$. \\[4pt]
  \textbf{Right panel (Lattice):} Plotted on separate axes because the
  lattice $E(t)$ decays \emph{exponentially} (reaches zero by $t\approx 1$),
  confirming $\alpha\to\infty$. If plotted on the same axes, it would
  compress all other curves to a thin band. \\[4pt]
  \textbf{Hierarchy:} Random $<$ quasicrystal $<$ crystal in terms of
  equilibration speed.
}

% ── Slide 13 ──────────────────────────────────────────────────────────
{
\setbeamercolor{frametitle}{bg=resultbg, fg=black}
\begin{frame}{\texorpdfstring{%
  Main Result: All Three Chains Converge to $\alpha=3$}{%
  Main Result: All Three Chains Converge to alpha=3}}
  \centering
  \includegraphics[width=0.78\linewidth]{fig8_alpha_extraction.png}
\end{frame}
}
\note{
  \textbf{This is the central result of the project.} \\[6pt]
  The effective exponent $\alpha(t)=-2\,d\ln E/d\ln t - 1$ is plotted
  versus diffusion time $t$.  Smoothing uses a period-aware sliding
  window matched to each chain's oscillation period $2\ln\mu$ in
  $\ln t$ space (cancels oscillatory contributions). \\[6pt]
  \textbf{Gray (Poisson):} Flat at $\alpha\approx 0$ --- correct. \\[4pt]
  \textbf{Black (Lattice):} Grows without bound --- $\alpha\to\infty$. \\[4pt]
  \textbf{Green/Purple/Red:} All plateau at $\alpha\approx 3$ within the
  gold-shaded measurement window $[10^2, 10^5]$. \\[6pt]
  \textbf{Measured values} (linear fit of $\ln E$ vs $\ln t$):
  Poisson = 0.001, Fibonacci = 3.049, Silver = 2.992, Bronze = 2.987.
  \\[6pt]
  All three metallic-mean chains confirm $\alpha=3$ via an independent
  dynamical measurement, matching the eigenvalue prediction.
}

% ── Slide 14: Results table ───────────────────────────────────────────
\begin{frame}{Extracted $\alpha$ Values}
  \begin{table}
    \centering
    \begin{tabular}{lccc}
      \toprule
      \textbf{Pattern} & $\boldsymbol{\alpha}$ \textbf{(measured)}
        & \textbf{Expected} & \textbf{Status}\\
      \midrule
      Poisson                  & $0.001$  & $0$       & \checkmark\\
      Integer Lattice          & exp.\ decay & $\infty$ & \checkmark\\[2pt]
      \textcolor{fibgreen}{Fibonacci (Golden Ratio)}
        & \textcolor{fibgreen}{\textbf{3.049}} & \textbf{3} & \checkmark\\
      \textcolor{silvpurp}{Silver Ratio}
        & \textcolor{silvpurp}{\textbf{2.992}} & \textbf{3} & \checkmark\\
      \textcolor{bronred}{Bronze Ratio}
        & \textcolor{bronred}{\textbf{2.987}}  & \textbf{3} & \checkmark\\
      \bottomrule
    \end{tabular}
  \end{table}

  \bigskip
  \begin{center}
    \fcolorbox{black}{resultbg}{\parbox{0.8\textwidth}{\centering
      \textbf{All three metallic-mean chains yield $\alpha\approx 3$},
      confirming the universal eigenvalue prediction via an independent
      dynamical measurement (diffusion spreadability).
    }}
  \end{center}
\end{frame}
\note{
  Values extracted via linear fit of $\ln E$ vs $\ln t$ over
  $[10^2, 10^5]$ (more robust than pointwise median, averages over
  oscillations).  Errors: Fibonacci 1.6\%, Silver 0.3\%, Bronze 0.4\%.
  These are consistent with finite-size effects from the FFT grid.

  The two benchmarks (Poisson $\alpha=0$ and lattice $\alpha\to\infty$)
  bracket the quasicrystal results and confirm the method is working
  correctly at both extremes.
}

% ======================================================================
\section{Technical Challenges}
% ======================================================================

% ── Slide 15 ──────────────────────────────────────────────────────────
\begin{frame}{Technical Challenges \& Solutions}
  \textbf{1.\ System size sensitivity (non-monotonic convergence):}
  \begin{itemize}\small
    \item FFT grid $k_n=2\pi n/L$ misaligns with irrational Bragg peaks
    \item Fibonacci ($\rho{=}0.72$, smallest $L$ for given $N$) needs
      $N\sim10^7$
    \item Convergence is \emph{non-monotonic}: $N{=}500\text{k}$ gives
      $\alpha{=}1.5$,\; $N{=}5.7\text{M}$ gives $\alpha{=}1.5$,\;
      $N{=}14.9\text{M}$ gives $\alpha{=}3.0$
  \end{itemize}

  \medskip
  \textbf{2.\ Integer lattice artifact:}
  \begin{itemize}\small
    \item FFT histogram binning creates noise floor mimicking $k^3$ $\;\Rightarrow\;$ spurious $\alpha\approx 3$
    \item \textbf{Fix:} analytical Bragg peak formula
      $E(t)=\tfrac{2\rho}{\phi_2}\sum_{n=1}^{50}|\tilde{m}(2\pi n)|^2
       e^{-(2\pi n)^2 Dt}$
  \end{itemize}

  \medskip
  \textbf{3.\ Noisy log derivative:}
  \begin{itemize}\small
    \item Point-by-point \texttt{np.gradient} amplifies Bragg peak noise
    \item \textbf{Fix 1:} Period-aware sliding window matched to
      oscillation period $2\ln\mu$ (cancels oscillatory contributions)
    \item \textbf{Fix 2:} Linear fit of $\ln E$ vs $\ln t$ over
      plateau window for single robust $\alpha$ value
  \end{itemize}
\end{frame}
\note{
  These challenges consumed the majority of the Phase~4 development time.
  The non-monotonic convergence was the most surprising finding: certain
  system sizes give \emph{worse} results than smaller ones because of how
  the FFT grid happens to align (or misalign) with the dominant Bragg
  peaks.

  Fibonacci is the hardest case because its high density ($\rho=0.72$)
  means the smallest domain $L$ for a given $N$, and hence the coarsest
  $k$-resolution.  Silver and Bronze converge at $N\sim 2$--$4$M.
}

% ======================================================================
\section{Summary \& Future Work}
% ======================================================================

% ── Slide 16 ──────────────────────────────────────────────────────────
\begin{frame}{Summary of All Results}
  \textbf{Real-space analysis (Phases 1--3):}
  \begin{itemize}\small
    \item Variance code validated against Poisson exact result (1.8\% error)
    \item Class~I hyperuniformity confirmed: bounded $\sigma^2(R)$ for all chains
    \item $\bar{\Lambda}$ increases with $\mu$:\;
      0.200 (Fib) $<$ 0.250 (Ag) $<$ 0.282 (Br)
    \item Rescaling invariance: substitution vs.\ projection agree to 0.1\%
    \item Class~I $\to$ II transition demonstrated via non-ideal strip width
  \end{itemize}

  \medskip
  \textbf{Reciprocal-space analysis (Phase 4):}
  \begin{itemize}\small
    \item $\tilde{\chi}_V(k)\sim k^3$ envelope confirmed for all chains
    \item Spreadability decay: Poisson $t^{-1/2}$,\;
      quasicrystal $t^{-2}$,\; lattice exp.
    \item \textbf{$\alpha = 3.0\pm 0.05$ for all three chains} (linear fit method)
  \end{itemize}

  \bigskip
  \begin{center}
    \fcolorbox{black}{resultbg}{\parbox{0.85\textwidth}{\centering
      $\alpha=3$ is a \textbf{universal property} of all metallic-mean
      1D substitution tilings, confirmed both \emph{analytically}
      (eigenvalue formula) and \emph{numerically} (spreadability).
    }}
  \end{center}
\end{frame}
\note{
  This slide summarizes all findings across four phases.

  The key narrative: we approached the same question ($\alpha=3$?) from
  two completely independent directions --- real-space variance analysis
  and reciprocal-space spreadability --- and both confirm the theoretical
  prediction.  The agreement across three distinct quasicrystal families
  with different metallic means provides strong evidence for universality.
}

% ── Slide 17 ──────────────────────────────────────────────────────────
\begin{frame}{Future Work: Phase 5 --- 2D Extensions}
  \textbf{Planned:}
  \begin{itemize}
    \item Construct 2D quasiperiodic tilings via the
      \textbf{Generalized Dual Method (GDM)}:
      \begin{itemize}
        \item 5-fold Penrose tiling (golden ratio) --- $\alpha=6$ known
        \item 8-fold octagonal tiling (silver mean) --- $\alpha$ \textbf{unknown}
        \item Bronze-ratio equivalent --- $\alpha$ \textbf{unknown}
      \end{itemize}
    \item Decorate vertices with disks ($\phi_2=0.25$)
    \item Compute 2D radial spectral density and extract $\alpha$
  \end{itemize}

  \bigskip
  \textbf{Open questions:}
  \begin{enumerate}
    \item Is $\alpha=6$ universal for all 2D metallic-mean tilings
      (analogous to $\alpha=3$ in 1D)?
    \item Does the eigenvalue formula generalize correctly to 2D
      inflation matrices?
    \item How large must 2D systems be for convergence?
  \end{enumerate}
\end{frame}
\note{
  The 2D Penrose tiling has $\alpha=6$ from the literature.
  The eigenvalue formula for 2D inflation matrices predicts
  $\alpha=6$ for Penrose, but the predictions for silver-mean and
  bronze-mean 2D tilings have not been tested numerically.

  Based on the 1D experience, system size requirements may be
  substantial and non-trivial.  The GDM provides a systematic way
  to generate arbitrary $n$-fold tilings.
}

% ── Slide 18: Backup ──────────────────────────────────────────────────
\appendix
\begin{frame}{Appendix: Code \& Reproducibility}
  \begin{table}
    \centering\small
    \begin{tabular}{ll}
      \toprule
      \textbf{File} & \textbf{Purpose}\\
      \midrule
      \texttt{substitution\_tilings.py}  & Chain generation (3 metallic means)\\
      \texttt{projection\_method.py}     & Cut-and-project from $\mathbb{Z}^2$\\
      \texttt{quasicrystal\_variance.py} & Number variance + $\bar{\Lambda}$\\
      \texttt{two\_phase\_media.py}      & Spectral density + spreadability\\
      \texttt{run\_all.py}               & Full pipeline: Figs 1--8 + tables\\
      \bottomrule
    \end{tabular}
  \end{table}

  \bigskip
  \textbf{Reproducibility:} \texttt{python run\_all.py}
  \begin{itemize}
    \item Deterministic (seed $=$ 42)
    \item Runtime: $\sim$18 minutes
    \item Generates all 8 figures + summary tables
    \item Dependencies: NumPy, Matplotlib, SciPy
  \end{itemize}
\end{frame}
\note{
  All code is self-contained in five Python files.  The full pipeline
  is run with a single command and is fully deterministic.

  The bottleneck is the Phase~4 FFT computation for the three
  quasicrystal chains at $N\sim 10^7$ (about 5 minutes each for
  Silver and Bronze due to larger FFT sizes).
}

\end{document}
