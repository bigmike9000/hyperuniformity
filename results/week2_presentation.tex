% !TEX program = pdflatex
\documentclass[aspectratio=169, 11pt]{beamer}

% -- Theme ------------------------------------------------------------------
\usetheme{Madrid}
\usecolortheme{seahorse}
\setbeamertemplate{navigation symbols}{}
\setbeamertemplate{footline}{%
  \leavevmode\hbox{%
    \begin{beamercolorbox}[wd=.33\paperwidth,ht=2.5ex,dp=1ex,center]{author in head/foot}%
      \usebeamerfont{author in head/foot}\insertshortauthor
    \end{beamercolorbox}%
    \begin{beamercolorbox}[wd=.34\paperwidth,ht=2.5ex,dp=1ex,center]{title in head/foot}%
      \usebeamerfont{title in head/foot}\insertshorttitle
    \end{beamercolorbox}%
    \begin{beamercolorbox}[wd=.33\paperwidth,ht=2.5ex,dp=1ex,right]{date in head/foot}%
      \usebeamerfont{date in head/foot}\insertframenumber{} / \inserttotalframenumber\hspace*{2ex}
    \end{beamercolorbox}}%
  \vskip0pt%
}

% -- Packages ---------------------------------------------------------------
\usepackage[T1]{fontenc}
\usepackage{lmodern}
\usepackage{amsmath,amssymb}
\usepackage{graphicx}
\usepackage{booktabs}
\usepackage{xcolor}
\usepackage{tikz}
\usepackage{pgfpages}
\usepackage{bookmark}

% -- Speaker notes ----------------------------------------------------------
\setbeameroption{show notes on second screen=right}

% -- Paths ------------------------------------------------------------------
\graphicspath{{./}}

% -- Custom colors ----------------------------------------------------------
\definecolor{fibgreen}{HTML}{2ca02c}
\definecolor{silvpurp}{HTML}{9467bd}
\definecolor{bronred}{HTML}{d62728}
\definecolor{copblue}{HTML}{1f77b4}
\definecolor{nickorg}{HTML}{ff7f0e}
\definecolor{resultbg}{HTML}{F0FFF0}
\definecolor{novelbg}{HTML}{FFFFF0}

% -- Title ------------------------------------------------------------------
\title[Hyperuniformity of 1D Point Patterns]{%
  Hyperuniformity of 1D Point Patterns}
\subtitle{Week 2 Progress: Expanding the Catalog}
\author{Michael Fang}
\institute{Princeton University}
\date{March 4, 2026}

% ======================================================================
\begin{document}

% -- Slide 1: Title --------------------------------------------------------
\begin{frame}
  \titlepage
\end{frame}
\note{
  Week 2 progress update. This week expanded the catalog from 4 patterns
  (lattice + 3 quasicrystals) to 25 patterns across all three
  hyperuniformity classes.
}

% ======================================================================
\section{Recap}
% ======================================================================

% -- Slide 2: Week 1 Recap ------------------------------------------------
\begin{frame}{Week 1 Recap}
  \textbf{Established:}
  \begin{itemize}
    \item Validated variance algorithm (Poisson benchmark, 1.1\% error)
    \item Generated Fibonacci, Silver, Bronze chains at $N\sim10^7$
    \item Confirmed $\alpha=3$ for all three via spreadability
      {\small(O\u{g}uz et al., 2019)}
    \item Computed $\bar{\Lambda}$ --- the surface-area coefficient
  \end{itemize}

  \bigskip
  \begin{table}
    \centering\small
    \begin{tabular}{lccc}
      \toprule
      \textbf{Pattern} & $\boldsymbol{\alpha}$ & $\boldsymbol{\bar{\Lambda}}$
        & \textbf{Class}\\
      \midrule
      Integer Lattice & $\infty$ & 1/6 & I\\
      Fibonacci & 3.05 & 0.200 & I\\
      Silver & 2.99 & 0.250 & I\\
      Bronze & 2.99 & 0.282 & I\\
      \bottomrule
    \end{tabular}
  \end{table}

  \bigskip
  \textbf{This week:} expand from 4 to \textbf{25 patterns}
  across all three hyperuniformity classes.
\end{frame}
\note{
  Brief recap of week 1. Key gap: only 4 patterns, only Class I,
  only $\alpha=3$ and $\alpha=\infty$.
}

% ======================================================================
\section{New Quasicrystals}
% ======================================================================

% -- Slide 3: Copper & Nickel Chains --------------------------------------
{
\setbeamercolor{frametitle}{bg=novelbg, fg=black}
\begin{frame}{\texorpdfstring{%
  Novel Result: Copper \& Nickel Quasicrystals}{%
  Novel Result: Copper and Nickel Quasicrystals}}
  Extended the metallic-mean family to $n=4$ (Copper) and $n=5$ (Nickel):

  \vspace{-0.5em}
  \begin{table}
    \centering\small
    \begin{tabular}{lcccccc}
      \toprule
      \textbf{Chain} & \textbf{Rule ($L\to$)} & $\boldsymbol{\mu_n}$
        & $\boldsymbol{N}$ & $\boldsymbol{\alpha}$ & $\boldsymbol{\bar{\Lambda}}$
        & \textbf{Literature}\\
      \midrule
      Fibonacci & $LS$ & 1.618 & 14.9M & 3.05 & 0.200 & 0.201$^a$\\
      Silver & $LLS$ & 2.414 & 22.6M & 2.99 & 0.250 & ---\\
      Bronze & $LLLS$ & 3.303 & 21.9M & 2.99 & 0.282 & ---\\
\textcolor{copblue}{\textbf{Copper}} & $LLLLS$ & 4.236 & 39.1M
        & \textcolor{copblue}{\textbf{3.03}} & \textcolor{copblue}{\textbf{0.293}} & \textbf{novel}\\
\textcolor{nickorg}{\textbf{Nickel}} & $LLLLLS$ & 5.193 & 16.4M
        & \textcolor{nickorg}{\textbf{2.88}} & \textcolor{nickorg}{\textbf{0.310}} & \textbf{novel}\\
      \bottomrule
    \end{tabular}
  \end{table}
  {\footnotesize $^a$Zachary \& Torquato (2009).}

  \vspace{-0.3em}
  \begin{columns}[T]
    \begin{column}{0.48\textwidth}
      \small
      \textbf{Key findings:}
      \begin{itemize}\setlength\itemsep{0.2em}
        \item $\alpha=3$ confirmed for all five chains
          {\footnotesize(eigenvalue formula: exact)}
        \item $\bar{\Lambda}$ \textbf{increases monotonically} with $n$
        \item $\Delta\bar{\Lambda}$: 0.050, 0.032, 0.011, 0.017
          $\Rightarrow$ converging toward a limit
      \end{itemize}
    \end{column}
    \begin{column}{0.48\textwidth}
      \small
      \textbf{Eigenvalue prediction} {\footnotesize(O\u{g}uz et al., 2019)}:
      \vspace{-0.5em}
      \begin{equation*}
        \alpha = 1 - \frac{2\ln|\lambda_2|}{\ln\lambda_1}
      \end{equation*}
      \vspace{-0.8em}
      For all metallic means, $\det M = -1$, so
      $|\lambda_2|=1/\lambda_1$ and $\alpha=3$ exactly.

      \vspace{0.2em}
      \fcolorbox{black}{novelbg}{\parbox{0.85\linewidth}{\centering\small
        Silver, Bronze, Copper, Nickel\\
        $\bar{\Lambda}$ values appear \textbf{novel}.
      }}
    \end{column}
  \end{columns}
\end{frame}
}
\note{
  Copper: $\mu_4 = 2+\sqrt{5}\approx 4.236$; $\bar\Lambda=0.293$.
  Nickel: $\mu_5 = (5+\sqrt{29})/2\approx 5.193$; $\bar\Lambda=0.310$.
  Both novel. Alpha confirmed via eigenvalue formula (exact 3.000)
  and spreadability fit (3.03, 2.88).
  The trend suggests $\bar{\Lambda}\to\text{limit}\lesssim 1/3$ as $n\to\infty$.
}

% ======================================================================
\section{Stealthy Patterns}
% ======================================================================

% -- Slide 4: Stealthy Hyperuniform ----------------------------------------
\begin{frame}{Stealthy Hyperuniform Patterns}
  \begin{columns}[T]
    \begin{column}{0.55\textwidth}
      \centering
      \includegraphics[width=\linewidth]{fig9_stealthy_sk.png}
    \end{column}
    \begin{column}{0.42\textwidth}
      \small
      \textbf{Data:} ${\sim}4{,}300$ configs/\,$\chi$\\
      {\footnotesize(Torquato group, $N{=}2{,}000$, $\rho{=}1$)}

      \smallskip
      $S(k)=0$ for $k < K(\chi)$, verified to $\sim\!10^{-14}$.

      \smallskip
      \begin{table}
        \centering\footnotesize
        \begin{tabular}{ccc}
          \toprule
          $\boldsymbol{\chi}$ & $\boldsymbol{\bar{\Lambda}}$ & \textbf{SEM}\\
          \midrule
          0.10 & 1.026 & 0.002\\
          0.20 & 0.528 & 0.001\\
          0.30 & 0.358 & 0.0004\\
          \bottomrule
        \end{tabular}
      \end{table}

      \smallskip
      \begin{itemize}\setlength\itemsep{0.1em}
        \item Class I, $\alpha\to\infty$
          {\footnotesize(exponential decay)}
        \item $\bar{\Lambda}$ \textbf{decreases} with $\chi$:
          more constrained $\Rightarrow$ more ordered
        \item No published $\bar{\Lambda}$ for comparison
      \end{itemize}
    \end{column}
  \end{columns}
\end{frame}
\note{
  Stealthy patterns: Torquato, Zhang, Stillinger (2015), Phys.\ Rev.\ X 5.
  The exclusion zone $S(k)=0$ for $k<K$ gives $\alpha\to\infty$ (faster than
  any power law). Lambda-bar from 500-config ensemble, SEM $<0.2\%$.
}

% ======================================================================
\section{Perturbed Lattices}
% ======================================================================

% -- Slide 5: Perturbed Lattice Overview -----------------------------------
\begin{frame}{Perturbed Lattice Models}
  Displace each site of $\mathbb{Z}$ by i.i.d.\ draw from distribution $f$
  {\small(Klatt, Kim, Torquato, 2020)}:

  \vspace{-0.3em}
  \begin{table}
    \centering\footnotesize
    \begin{tabular}{llccl}
      \toprule
      \textbf{Distribution} & $\boldsymbol{\hat{f}(k)}$ & $\boldsymbol{\alpha}$
        & \textbf{Class} & \textbf{Parameters}\\
      \midrule
      Uniform $[-a/2,\,a/2)$ & $\operatorname{sinc}(ka/2\pi)$
        & 2 & I & $a=0.1$--$1.0$\\
      Gaussian $\mathcal{N}(0,\sigma^2)$ & $e^{-\sigma^2k^2/2}$
        & 2 & I & $\sigma=0.1$--$0.5$\\
      Cauchy$(0,\gamma)$ & $e^{-\gamma|k|}$
        & 1 & II & $\gamma=0.1$\\
      Stable$(s,c)$ & $e^{-c^s|k|^s}$
        & $s$ & I/II/III & $s=0.3$--$1.7$\\
      \bottomrule
    \end{tabular}
  \end{table}

  \vspace{-0.8em}
  \centering
  \includegraphics[width=0.42\linewidth]{fig10_perturbed_variance.png}
\end{frame}
\note{
  Perturbed lattices: Klatt, Kim, Torquato (2020), Phys.\ Rev.\ E 101, 032118.
  The displacement distribution's characteristic function $\hat{f}(k)$ controls $\alpha$:
  $S(k)\sim 1-|\hat{f}(k)|^2$ near $k=0$.
  URL and Gaussian give $\alpha=2$ (Class I), Cauchy gives $\alpha=1$ (Class II),
  stable with $s<1$ gives Class III.
}

% -- Slide 6: URL Exact Formula -------------------------------------------
\begin{frame}{\texorpdfstring{%
  URL Model: Exact $\bar{\Lambda}(a)$ Validated}{%
  URL Model: Exact Lambda-bar Validated}}
  \begin{columns}[T]
    \begin{column}{0.58\textwidth}
      \centering
      \includegraphics[width=\linewidth]{fig14_url_lambda_curve.png}
    \end{column}
    \begin{column}{0.38\textwidth}
      \small
      \textbf{Exact formula} {\footnotesize(Klatt et al., 2020, Eq.~B7)}:
      \vspace{-0.5em}
      \begin{equation*}
        \bar{\Lambda}(a) = \frac{a}{3} + \frac{\{a\}^2(1{-}\{a\})^2}{6a^2}
      \end{equation*}
      \vspace{-0.8em}
      where $\{a\}=a-\lfloor a\rfloor$.

      \medskip
      \begin{table}
        \centering\footnotesize
        \begin{tabular}{cccc}
          \toprule
          $\boldsymbol{a}$ & \textbf{Num.} & \textbf{Exact} & \textbf{Err}\\
          \midrule
          0.1 & 0.168 & 0.168 & 0.2\%\\
          0.3 & 0.181 & 0.182 & 0.2\%\\
          0.5 & 0.208 & 0.208 & 0.1\%\\
          0.8 & 0.274 & 0.273 & 0.1\%\\
          1.0 & 0.333 & 0.333 & 0.0\%\\
          \bottomrule
        \end{tabular}
      \end{table}

      \smallskip
      \fcolorbox{black}{resultbg}{\parbox{0.9\linewidth}{\centering\small
        All 5 points match exact\\
        curve within \textbf{0.2\%}.
      }}
    \end{column}
  \end{columns}
\end{frame}
\note{
  The URL (Uniformly Randomized Lattice) has an exact $\bar{\Lambda}(a)$ formula
  from Klatt et al.\ (2020), Eq.~B7. At $a=0$: lattice ($1/6$).
  At $a=1$: cloaked ($1/3$, all Bragg peaks vanish).
  Our numerics at $N=100{,}000$ match to $<0.2\%$ for all $a$ values.
}

% ======================================================================
\section{Filling the Alpha Gap}
% ======================================================================

% -- Slide 7: Stable s>1 --------------------------------------------------
\begin{frame}{\texorpdfstring{%
  Filling the $1<\alpha<2$ Gap: Stable Perturbations}{%
  Filling the Alpha Gap: Stable Perturbations}}
  \begin{columns}[T]
    \begin{column}{0.55\textwidth}
      \centering
      \includegraphics[width=\linewidth]{fig13_stable_classI_variance.png}

      \vspace{-0.3em}
      {\footnotesize Bounded variance confirms Class~I for all $s>1$.}
    \end{column}
    \begin{column}{0.42\textwidth}
      \small
      Symmetric stable displacements with index $s>1$:\\
      $\hat{f}(k) = e^{-c^s|k|^s} \;\Rightarrow\; \alpha = s$.

      \medskip
      \begin{table}
        \centering\footnotesize
        \begin{tabular}{cccc}
          \toprule
          $\boldsymbol{s}$ & $\boldsymbol{\alpha}$ (fit)
            & $\boldsymbol{\bar{\Lambda}}$ & \textbf{Class}\\
          \midrule
          1.3 & 1.47 & 0.417 & I\\
          1.5 & 1.75 & 0.303 & I\\
          1.7 & 1.91 & 0.242 & I\\
          \bottomrule
        \end{tabular}
      \end{table}

      \medskip
      \textbf{Before:} only $\alpha=2$ (Gaussian/URL)\\
      and $\alpha=3$ (quasicrystals) in Class~I.

      \medskip
      \textbf{Now:} continuous family of Class~I\\
      patterns with $1 < \alpha < 2$.

      \medskip
      {\footnotesize CMS algorithm for stable RVs;\\
      scale $c=0.1$, $N{=}100{,}000$.}
    \end{column}
  \end{columns}
\end{frame}
\note{
  Stable distributions with $s>1$ give Class I hyperuniform patterns
  with $\alpha=s$, filling the gap $1<\alpha<2$.
  Chambers-Mallows-Stuck algorithm generates symmetric stable RVs.
  The same code handles $s<1$ (Class III) and $s>1$ (Class I).
}

% ======================================================================
\section{Comprehensive Ranking}
% ======================================================================

% -- Slide 8: Ranking Chart (Main Result) ----------------------------------
{
\setbeamercolor{frametitle}{bg=resultbg, fg=black}
\begin{frame}{\texorpdfstring{%
  Main Result: 25-Pattern $(\alpha,\,\bar{\Lambda})$ Ranking}{%
  Main Result: 25-Pattern Ranking}}
  \centering
  \includegraphics[width=0.95\linewidth]{fig11_ranking.png}
\end{frame}
}
\note{
  The comprehensive ranking chart: 25 patterns sorted by $\bar{\Lambda}$.
  Left panel: log scale shows all classes.
  Right panel: linear scale zooms into Class I.
  Lower $\bar{\Lambda}$ = more ordered.
  Integer lattice is the most ordered ($\bar{\Lambda}=1/6$).
  Stable $s=0.3$ is the least ordered Class III pattern ($\bar{\Lambda}\sim 73$).
}

% -- Slide 9: Ranking Table -----------------------------------------------
\begin{frame}{\texorpdfstring{%
  Class I Ranking: $\bar{\Lambda}$ from Lattice to Disorder}{%
  Class I Ranking}}
  \begin{table}
    \centering\small
    \begin{tabular}{rlccc}
      \toprule
      \textbf{\#} & \textbf{Pattern} & $\boldsymbol{\alpha}$
        & $\boldsymbol{\bar{\Lambda}}$ & \textbf{Category}\\
      \midrule
       1 & Integer Lattice & $\infty$ & 0.167 & crystal\\
       2 & URL $a{=}0.1$ & 2.0 & 0.168 & perturbed\\
       3 & Gaussian $\sigma{=}0.1$ & 2.0 & 0.187 & perturbed\\
       4 & Fibonacci & 3.0 & 0.200 & quasicrystal\\
       5 & URL $a{=}0.5$ & 2.0 & 0.208 & perturbed\\
       6 & Stable $s{=}1.7$ & 1.9 & 0.242 & perturbed\\
       7 & Silver & 3.0 & 0.250 & quasicrystal\\
 8 & \textbf{Copper} & \textbf{3.0} & \textbf{0.293} & quasicrystal\\
 9 & \textbf{Nickel} & \textbf{2.9} & \textbf{0.310} & quasicrystal\\
      10 & URL $a{=}1.0$ (cloaked) & 2.0 & 0.333 & perturbed\\
      11 & Stealthy $\chi{=}0.3$ & $\infty$ & 0.358 & stealthy\\
      12 & Stealthy $\chi{=}0.1$ & $\infty$ & 1.026 & stealthy\\
      \midrule
      \multicolumn{5}{c}{\textit{+ Cauchy (Class II) and 3 stable $s{<}1$ (Class III)}}\\
      \bottomrule
    \end{tabular}
  \end{table}

  \vspace{-0.8em}
  {\footnotesize
  Lower $\bar{\Lambda}$ = more ordered.
  Lattice is ground state; quasicrystals intermediate;
  stealthy (despite $\alpha\!\to\!\infty$) have large $\bar{\Lambda}$
  from residual short-range disorder.}
\end{frame}
\note{
  Key insight: $\alpha$ and $\bar{\Lambda}$ are complementary metrics.
  Stealthy patterns have $\alpha\to\infty$ but $\bar\Lambda\gg 1/6$,
  while quasicrystals have finite $\alpha=3$ but smaller $\bar\Lambda$.
  This shows why both metrics are needed for a complete ranking.
}

% ======================================================================
\section{Validation \& Open Questions}
% ======================================================================

% -- Slide 10: Literature Validation --------------------------------------
\begin{frame}{Literature Validation}
  \begin{columns}[T]
    \begin{column}{0.48\textwidth}
      \small
      \textbf{$\bar{\Lambda}$ comparisons:}
      \begin{table}
        \centering\footnotesize
        \begin{tabular}{lccc}
          \toprule
          \textbf{Pattern} & \textbf{Ours} & \textbf{Lit.} & \textbf{Err}\\
          \midrule
          Lattice & 0.167 & $1/6$ & exact\\
          Fibonacci & 0.200 & 0.201$^a$ & 0.5\%\\
          URL $a{=}1$ & 0.333 & $1/3$ $^b$ & 0.0\%\\
          URL $a{=}0.5$ & 0.208 & 0.208$^b$ & 0.1\%\\
          \midrule
          Silver & 0.250 & --- & \textit{novel}\\
          Bronze & 0.282 & --- & \textit{novel}\\
          Copper & 0.293 & --- & \textit{novel}\\
          Nickel & 0.310 & --- & \textit{novel}\\
          \bottomrule
        \end{tabular}
      \end{table}
      {\footnotesize $^a$Zachary \& Torquato (2009).\\
       $^b$Klatt et al.\ (2020), Eq.~B7.}
    \end{column}
    \begin{column}{0.48\textwidth}
      \small
      \textbf{$\alpha$ comparisons:}
      \begin{table}
        \centering\footnotesize
        \begin{tabular}{lccc}
          \toprule
          \textbf{Pattern} & \textbf{Ours} & \textbf{Expected} & \textbf{Err}\\
          \midrule
          Fibonacci & 3.05 & 3 & 1.6\%\\
          Silver & 2.99 & 3 & 0.3\%\\
          Bronze & 2.99 & 3 & 0.4\%\\
          Copper & 3.03 & 3 & 0.9\%\\
          Nickel & 2.88 & 3 & 4.0\%\\
          \midrule
          URL (all $a$) & ${\approx}2.0$ & 2 & $<$4\%\\
          Cauchy & 0.95 & 1 & 4.9\%\\
          \bottomrule
        \end{tabular}
      \end{table}

      \medskip
      \fcolorbox{black}{resultbg}{\parbox{0.9\linewidth}{\centering\small
        All published values\\
        reproduced within \textbf{5\%}.
      }}
    \end{column}
  \end{columns}
\end{frame}
\note{
  All $\alpha$ and $\bar{\Lambda}$ values validated against published results.
  Fibonacci $\bar{\Lambda}=0.201$ from Zachary \& Torquato (2009), Table 1.
  URL exact formula from Klatt et al.\ (2020), Eq.~B7.
  Eigenvalue prediction from O\u{g}uz et al.\ (2019).
  Silver, Bronze, Copper, Nickel $\bar{\Lambda}$ values appear to be novel.
}

% -- Slide 11: Open Questions ---------------------------------------------
\begin{frame}{Open Questions}
  \textbf{1. The $2 < \alpha < 3$ gap:}
  \begin{itemize}
    \item All metallic-mean substitutions give $\alpha=3$ exactly ($\det M=-1$)
    \item All finite-variance perturbations give $\alpha=2$ exactly
    \item \textbf{No known 1D construction achieves $2<\alpha<3$}
    \item Is this a fundamental gap, or does a construction exist?
  \end{itemize}

  \bigskip
  \textbf{2. Metallic-mean $\bar{\Lambda}$ convergence:}
  \begin{itemize}
    \item $\bar{\Lambda}(n)$: 0.200, 0.250, 0.282, 0.293, 0.310
    \item Successive differences decreasing --- what is $\bar{\Lambda}_\infty$?
    \item Is the limit $1/3$ (the cloaked-URL value)?
  \end{itemize}

  \bigskip
  \textbf{3. Next steps:}
  \begin{itemize}
    \item Add period-doubling chains (Class II, $\alpha=1$)
    \item Obtain more stealthy $\chi$ values from grad student
    \item Begin writing JP paper with the 25-pattern ranking table
  \end{itemize}
\end{frame}
\note{
  The $2<\alpha<3$ gap is a genuine open problem.
  Non-Pisot substitution matrices could potentially give $2<\alpha<3$,
  but these are rare and may not produce quasicrystalline patterns.
  The $\bar{\Lambda}$ convergence question could be addressed analytically
  or by computing $n=6,7,\ldots$ chains.
}

\end{document}
