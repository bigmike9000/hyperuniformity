\documentclass[12pt]{article}
\usepackage[margin=1in]{geometry}
\usepackage{amsmath, amssymb}
\usepackage{parskip}

\title{A Comprehensive Ranking of Hyperuniform One-Dimensional Point Patterns}
\author{Michael Fang \\
\small Advisor: Professor Salvatore Torquato \\
\small Department of Physics, Princeton University}
\date{Spring 2026}

\begin{document}
\maketitle
\thispagestyle{empty}

Hyperuniform point patterns, characterized by anomalously suppressed large-scale density fluctuations, arise across a diverse range of physical contexts from quasicrystals to disordered packings. Despite substantial theoretical progress, the existing catalog of characterized one-dimensional hyperuniform structures remains sparse, particularly for Class~I systems ($\alpha > 1$). This project aims to provide a much more comprehensive ranking of 1D hyperuniform point configurations by systematically computing the hyperuniformity exponent~$\alpha$ and surface-area coefficient~$\bar{\Lambda}$ across quasicrystals and exotic disordered systems. 

A key motivation for this project is that hyperuniform systems often exhibit unusual physical properties, including anomalous transport, photonic band gaps, and mechanical rigidity that correlate with their structural metrics. Establishing a comprehensive $(\alpha, \bar{\Lambda})$ ranking also lays groundwork for making these connections more precise.

The paper will begin by introducing hyperuniformity and its two complementary characterizations: the real-space number variance $\sigma^2(R)$ and the reciprocal-space structure factor scaling $S(k) \sim |k|^\alpha$. We will go over the classification into Classes~I--III via the exponent~$\alpha$ and the role of the surface-area coefficient~$\bar{\Lambda}$ as a measure of order within each class. Next, we will describe the computational methodology, including pattern generation via substitution and projection methods for quasicrystals, as well as algorithms for producing disordered hyperuniform configurations such as stealthy hyperuniform patterns and perturbed lattices. We will also discuss how the number variance and exponent~$\alpha$ are computed through the sliding-window method and diffusion spreadability. 

Preliminary results on the three metallic-mean quasicrystals (Fibonacci, the silver ratio, and the bronze ratio chains) have confirmed Class~I hyperuniformity with the theoretically predicted universal exponent $\alpha = 3$, as well as given the expected ordering $\bar{\Lambda}_{\mathrm{Fib}} < \bar{\Lambda}_{\mathrm{Ag}} < \bar{\Lambda}_{\mathrm{Br}}$. Next steps in the project are to expand our results to include disordered hyperuniform systems, like stealthy hyperuniform point patterns and perturbed lattices, and to compile a unified $(\alpha, \bar{\Lambda})$ ranking table. The paper will conclude by discussing what this ranking reveals about the landscape of order and disorder in one dimension, and the extent to which $\alpha$ and $\bar{\Lambda}$ together distinguish structurally distinct systems.

\end{document}
